%Title:
%Started:
%Updated:
\documentclass{amsart}
\usepackage{amssymb,latexsym,amsmath}
\usepackage{graphicx}
\usepackage{epsfig}
\usepackage{color}
\usepackage{enumerate}
\usepackage{hyperref}
%-----------------------------------------------------------------
\vfuzz2pt % Don't report over-full v-boxes if over-edge is small
\hfuzz2pt % Don't report over-full h-boxes if over-edge is small
% THEOREMS -------------------------------------------------------
\theoremstyle{plain}
\newtheorem{thm}{Theorem}
\newtheorem{cor}{Corollary}
\newtheorem{lem}{Lemma}
\newtheorem{prop}{Proposition}
\theoremstyle{definition}
\newtheorem{defn}{Definition}
\theoremstyle{remark}
\newtheorem{rem}{Remark}
\theoremstyle{definition}
\newtheorem{ex}{Example}
\numberwithin{equation}{section}
\newtheorem{prob}{Problem}
\numberwithin{equation}{section}
% Colors-----------------------------------------------------------
\definecolor{Green}{rgb}{0,.5,0}
%use for definitions
\definecolor{Red}{rgb}{.8,.2,0}
%use for emphasis
\definecolor{Yellow}{rgb}{.6,.6,.1}
%use for part titles
\definecolor{Cyan}{rgb}{.2,.6,.7}
%use for comments
\definecolor{Purple}{rgb}{.4,0,1}
%use for examples
\definecolor{deepred}{rgb}{.53,.29,.24}
%use for important points
\definecolor{Black}{rgb}{0,0,0}
%use for washout
\definecolor{Grey}{rgb}{.45,.45,.45}
% use for theorems
\newcommand{\tred}[1]{\textcolor{Red}{#1}}
\newcommand{\tgreen}[1]{\textcolor{Green}{#1}}
\newcommand{\tcyan}[1]{\textcolor{Cyan}{#1}}
\newcommand{\tyellow}[1]{\textcolor{Yellow}{#1}}
\newcommand{\tpurple}[1]{\textcolor{Purple}{#1}}
\newcommand{\tblack}[1]{\textcolor{Black}{#1}}
\newcommand{\tgrey}[1]{\textcolor{Grey}{#1}}
\newcommand{\tdeepred}[1]{\textcolor{deepred}{#1}}
\newcommand{\ttt}[1]{\texttt{#1}}
% MATH -----------------------------------------------------------
\newcommand{\norm}[1]{\left\Vert#1\right\Vert}
\newcommand{\abs}[1]{\left\vert#1\right\vert}
\newcommand{\set}[1]{\left\{#1\right\}}
\newcommand{\Real}{\mathbb R}
\newcommand{\eps}{\varepsilon}
\newcommand{\To}{\longrightarrow}
\newcommand{\BX}{\mathbf{B}(X)}
\newcommand{\A}{\mathcal{A}}
\newcommand{\lv}{\left\langle}
\newcommand{\rv}{\right\rangle}
\newcommand{\mbf}[1]{\mathbf{#1}}
\newcommand{\mat}[2][rrrrrrrrrrrrrrrrrrrrrrrrr]{\left(\begin{array}{#1}#2\\ \end{array}\right)}
% ----------------------------------------------------------------
\setlength{\topmargin}{-.3in}
\setlength{\headheight}{.2in}
\setlength{\headsep}{.3in}
\setlength{\oddsidemargin}{0in}
\setlength{\evensidemargin}{0in}
\setlength{\textwidth}{6.5in}
\setlength{\textheight}{8.5in}
\renewcommand{\baselinestretch}{1.5}
% ----------------------------------------------------------------

\begin{document}

\title{Instructions for setting up your laptop}
\author{Troy Butler and Varis Carey}

\maketitle

\section*{Step 1: Obtaining a distribution of Python 3.6}

Go to \href{https://www.continuum.io/downloads}{https://www.continuum.io/downloads} and follow the instructions to download and install a Python 3.6 distribution for whichever operating system your laptop has. 
The Python package manager installed is called Anaconda. 

\section*{Step 2: Testing the Jupyter Notebook}

\subsection*{For Windows:}

In the apps/start menu, you should see a folder labeled \verb|Anaconda3 (64-bit)| (or \verb|Anaconda3 (32-bit)| if you are using an older laptop and installed the 32-bit version). 
Left click on that folder and left click on the \verb|Jupyter Notebook| icon.
A command terminal titled \verb|Jupyter Notebook| should open up followed by your default web browser opening up a new tab displaying a URL starting with \verb|http://localhost:8888/tree| and \verb|jupyter| should be displayed at the top of the webpage shown in this tab. 
You should see many folder options for you to navigate through such as \verb|Desktop| and \verb|Documents| among others. 
For now, close this tab in the web browser and enter Ctrl+C twice in the command terminal to close everything down.

Congratulations, everything is working properly. 

\subsection*{For Mac OS or Linux:}

Open up a terminal and type \verb|jupyter notebook|. 
Your default web browser should open up a new tab showing a URL starting with \verb|http://localhost:8888/tree| and \verb|jupyter| should be displayed at the top of the webpage. 
Whatever directory the terminal was set to when you typed \verb|jupyter notebook| will have all of its contents displayed on the webpage. 
For now, close the tab in the web browser associated with the notebook and enter Ctrl+C twice in the terminal to close everything down.

Congratulations, everything is working properly. 

\section*{Step 3: Obtain a copy of the lectures}

Go to \href{https://github.com/variscarey/CCM-Intro-to-SC}{https://github.com/variscarey/CCM-Intro-to-SC} and either clone or download the repository in the main branch.
If you are unfamiliar with using \verb|git|, then just download the repository and unzip it to wherever you want the repository to reside on your hard drive. 
We recommend putting the directory on your desktop for ease of access. 




\end{document}
